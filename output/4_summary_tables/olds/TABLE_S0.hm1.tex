\begin{table}

\caption{\label{tab:}Cooperative and non-cooperative virulence factors in the global pool of coding sequences.
               Corresponding odds ratio (OR) and chi-square test testing whether the OR is significantly greater than 1.
               The odds ratio is as defined in the main text methods: ratio of the odds of a virulence factor being cooperative to the odds of a non-virulence factor being cooperative.
               Abbreviations: VF = Virulence Factor, non-VF = not virulence factor, Coop = cooperative, non-Coop = not cooperative, QS = quorum-sensing, All = when a coding sequence is considered cooperative if it is annotated as any of the 6 forms of cooperation.}
\centering
\begin{tabular}[t]{lrrrrrrrrrl}
\toprule
Trait & (VF,Coop) & (VF,non-Coop) & (non-VF,Coop) & (non-VF,non-Coop) & OR & l-95\% CI & u-95\% CI & Chisq & df & pvalue\\
\midrule
Secretome & 122 & 4600 & 4391 & 358049 & 2.16 & 1.80 & 2.59 & 71.16 & 1 & 3.3e-17\\
Biofilm & 48 & 864 & 4465 & 361785 & 4.50 & 3.36 & 6.03 & 119.24 & 1 & 9.3e-28\\
Siderophore & 14 & 542 & 4499 & 362107 & 2.08 & 1.22 & 3.54 & 6.59 & 1 & 0.01\\
QS & 9 & 151 & 4504 & 362498 & 4.80 & 2.45 & 9.40 & 21.98 & 1 & 2.7e-06\\
Antib. degr. & 12 & 522 & 4501 & 362127 & 1.85 & 1.04 & 3.28 & 3.76 & 1 & 0.05\\
\addlinespace
Secr. Syst. & 27 & 448 & 4486 & 362201 & 4.87 & 3.30 & 7.18 & 74.12 & 1 & 7.3e-18\\
All & 229 & 6859 & 4284 & 355790 & 2.77 & 2.42 & 3.17 & 236.85 & 1 & 1.9e-53\\
\bottomrule
\end{tabular}
\end{table}
